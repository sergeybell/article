%------------------------------------------------
\section{Заключение}

%------------------------------------------------
\begin{frame}
	\frametitle{Основные параметры структур}

\begin{table}
\begin{center}
\begin{tabular}{|l|c|c|c|} 
\hline
Структура & Регулярная & Резонансная & Комбинир. \\
\hline Частицы & ${ }_{79}^{97} \mathrm{Au}$ & $\mathrm{p}$, $\mathrm{~d}$ & $\mathrm{p}$, $\mathrm{~d}$ \\
\hline Энергия эксперимента, ГэВ/нуклон & 4.5 & 12.6 & 12.6 \\
\hline Критическая энергия $\gamma_{\mathrm{tr}}$  & 7 & 15 & $i$50 \\
\hline Глубина модуляции & - & 25 \% & 45 \% \\
\hline Время охлаждения при 4.5 ГэВ, c & 2500 & 1500 & 800 \\
\hline Время ВПР при 4.5 ГэВ, c & 2500 & 400 & 250 \\
\hline
\end{tabular}
\end{center}
\label{tab:dual}
\end{table}

\end{frame}

\begin{frame}
	\frametitle{Заключение}
	
\par Гибкость дуальной структуры заключается в сочетании подходов, позволяющих обеспечивать контроль над ВПР для тяжёлых частиц и стабилизировать пучок при переходе через критическую энергию для лёгких частиц. 
\newline
\par Такой подход делает структуру универсальной для проведения коллайдерных экспериментов.
	
\end{frame}

\begin{frame}
	\frametitle{Благодарность}
	\begin{center}
		Спасибо за внимание!
	\end{center}
	
\end{frame}
