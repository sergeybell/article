% !TeX spellcheck = en_GB
% !TeX program = lualatex
%
% v 2.3  Feb 2019   Volker RW Schaa
%		# changes in the collaboration therefore updated file "jacow-collaboration.tex"
%		# all References with DOIs have their period/full stop before the DOI (after pp. or year)
%		# in the author/affiliation block all ZIP codes in square brackets removed as it was not %         understood as optional parameter and ZIP codes had bin put in brackets
%       # References to the current IPAC are changed to "IPAC'19, Melbourne, Australia"
%       # font for ‘url’ style changed to ‘newtxtt’ as it is easier to distinguish "O" and "0"
%
\documentclass[a4paper,
               %boxit,        % check whether paper is inside correct margins
               %titlepage,    % separate title page
               %refpage       % separate references
               %biblatex,     % biblatex is used
               keeplastbox,   % flushend option: not to un-indent last line in References
               %nospread,     % flushend option: do not fill with whitespace to balance columns
               %hyphens,      % allow \url to hyphenate at "-" (hyphens)
               %xetex,        % use XeLaTeX to process the file
               %luatex,       % use LuaLaTeX to process the file
               ]{jacow}
%
% ONLY FOR \footnote in table/tabular
%
\usepackage{pdfpages,multirow,ragged2e} %
%%% Работа с русским языком
\usepackage[T2A]{fontenc}			% кодировка
\usepackage[utf8]{inputenc}			% кодировка исходного текста
\usepackage[english,russian]{babel}
\renewcommand\abstractname{GG}

%
% CHANGE SEQUENCE OF GRAPHICS EXTENSION TO BE EMBEDDED
% ----------------------------------------------------
% test for XeTeX where the sequence is by default eps-> pdf, jpg, png, pdf, ...
%    and the JACoW template provides JACpic2v3.eps and JACpic2v3.jpg which
%    might generates errors, therefore PNG and JPG first
%
\makeatletter%
	\ifboolexpr{bool{xetex}}
	 {\renewcommand{\Gin@extensions}{.pdf,%
	                    .png,.jpg,.bmp,.pict,.tif,.psd,.mac,.sga,.tga,.gif,%
	                    .eps,.ps,%
	                    }}
\makeatother

% CHECK FOR XeTeX/LuaTeX BEFORE DEFINING AN INPUT ENCODING
% --------------------------------------------------------
%   utf8  is default for XeTeX/LuaTeX
%   utf8  in LaTeX only realises a small portion of codes
%
\ifboolexpr{bool{xetex} or bool{luatex}} % test for XeTeX/LuaTeX
 {}                                      % input encoding is utf8 by default
%{\usepackage[utf8]{inputenc}}           % switch to utf8
%\usepackage[USenglish]{babel}

%
% if BibLaTeX is used
%
\ifboolexpr{bool{jacowbiblatex}}%
 {%
  \addbibresource{jacow-test.bib}
  \addbibresource{biblatex-examples.bib}
 }{}
\listfiles


%%
%%   Lengths for the spaces in the title
%%   \setlength\titleblockstartskip{..}  %before title, default 3pt
%%   \setlength\titleblockmiddleskip{..} %between title + author, default 1em
%%   \setlength\titleblockendskip{..}    %afterauthor, default 1em

\begin{document}

\title{Продольная динамика в ВЧ барьерного типа при критической энергии, включая импедансы в BLonD в NICA.}

\author{С.Колокольчиков\textsuperscript{1, 2}\thanks{email: sergey.bell13@gmail.com}, Ю. Сеничев\textsuperscript{1, 2}, А. Аксентьев\textsuperscript{1, 2, 3}, \\ А. Мельников\textsuperscript{1, 2, 4}, Е. Сыресин\textsuperscript{5}, В. Ладыгин\textsuperscript{5}\\
	\textsuperscript{1}Институт ядерных исследований (РАН), Москва, Россия,\\
	\textsuperscript{2}Московский физико-технического института (НИУ), Долгопрудный, Россия,\\
	\textsuperscript{3}Московский инженерно-физический институт (НИУ), Москва, Россия,\\
	\textsuperscript{4}Институт теоретической физики им. Л.Д. Ландау, Черноголовка, Россия,\\
	\textsuperscript{5}Объединенный институт ядерных исследований, Дубна, Россия\\}

\maketitle

\section{\abstractname}
\par В статье исследуется влияние импедансов пространственного заряда, а также ВЧ на продольную динамику во время процедуры преодоления критической энергии скачком. Отличительной особенностью является использование ВЧ барьерного типа, в результате чего достигается специфическое распределение пучка в фазовом пространстве, отличное от классического, формируемого гармоническим ВЧ.

\section{Критическая энергия}

\par При рассмотрении продольного движения вводится понятие коэффициента расширения орбиты (momentum compaction factor) \cite{Lee}:

\begin{equation}
\alpha_c=\frac{1}{R_0} \frac{d R}{d \delta}=\alpha_0+2 \alpha_1 \delta+3 \alpha_2 \delta^2+\cdots \equiv \frac{1}{\gamma_T^2}
\label{alpha}
\end{equation}

\par и коэффициента скольжения (slip-factor):

\begin{equation}
\eta(\delta)=-\frac{1}{\omega_0} \frac{\Delta \omega}{\delta}=-\left(\eta_0+\eta_1 \delta+\eta_2 \delta^2+\cdots\right),
\label{eta}
\end{equation}

где $\delta$ – разброс по импульсам, $R_{0},R$ – усреднённый радиус референсной и отклоненной на $\delta$ частиц, $\omega, \omega_{0}$ – соответствующие частоты, $\alpha_n, \eta_n$ – n-ые члены разложения, $\gamma_{tr}$ – критическая энергия. Коэффициенты могут быть связаны соотношениями $\eta=\eta_{0}=\alpha_{0}-\frac{1}{\gamma_{0}^{2}}$, $\eta_{1}=\alpha_{1}-\frac{\eta_{0}}{\gamma_{0}^{2}}+\frac{3}{2} \frac{\beta^{2}}{\gamma^{2}}$ . Как видно при определённой энергии референсной частицы – критической $\gamma = \gamma_{tr}$, коэффициент скольжения принимает нулевое значение $\eta = \eta_{0} = 0$.

\section{Скачок критической энергии}

\par Процедура скачка критической энергии применяется для преодоления критической энергии. Таким образом, удается сохранить устойчивое движение пучка в фазовом пространстве. Данный метод применялся на многих установках и описан в работах \cite{tr}, \cite{tr_pip}.

\par Необходимость скачка можно понять, рассмотрев зависимость от $\eta(\delta)=\eta_{0}+\eta_{1} \delta+\cdots$, уравнений продольного движения, которые описывают эволюцию частиц в фазовом пространстве \cite{BB_th}:

\begin{equation}
\begin{aligned}
& \frac{d \tau}{d t}=\eta(\delta) \cdot \frac{h \cdot \Delta E}{\beta^2 \cdot E_0} \\
& \frac{d(\Delta E)}{d t}=\frac{V(\tau)}{T_0}
\end{aligned}
\label{long}
\end{equation}

\par При ускорении, значение коэффициента скольжения $\eta$ приближается к нулю для всех частиц, однако из-за ненулевого разброса по импульсам $\delta$, слагаемое $\eta_1\delta$ начинает быть сравнимо с $\eta_0$ и играет важную роль на динамику вблизи критической энергии. Если не предпринимать никаких мер, то для частиц, преодолевших критическую энергию, знак коэффициента скольжения меняется. Исходя из Уравнений \ref{long}, видно, что движение в фазовой плоскости становится не устойчивым ведёт к потере пучка. Процедура скачка позволяет, во-первых, в течение поднятия критической энергии, удерживать пучок на расстоянии, достаточном, чтобы все частицы имели один и тот же знак коэффициента скольжения. Во-вторых, обеспечить быстрый переход к новому состоянию, где коэффициент скольжения меняет знак, но для всех частиц снова имеет одинаковый знак. Стабильность обеспечивается сменой полярности удерживающих ВЧ-барьеров.

 Для коэффициента расширения орбиты может быть получено выражение \cite{resonant}:
 
\begin{equation}
\alpha=\frac{1}{C}\int_{0}^{C}{\frac{D\left(s\right)}{\rho\left(s\right)}ds}
\label{alpha_c}
\end{equation}

где $D\left(s\right)$– дисперсионная функция, $1/\rho\left(s\right)$ – кривизна орбиты. Для стационарной машины, возможно вариация дисперсионной функции для изменения значения $\alpha$, а соответственно и $\eta$. Например, для NICA, рассматривается возможность создания дополнительного градиента в квадрупольных линзах. Расчёты показывают, что возможно изменение критической энергии $\gamma_{tr}$ со скоростью $d\gamma_{tr}/\ dt\ =\ 8.5\ c^{-1}$ \cite{syresin}.

Можно выделить пять основных состояний продольной динамики, основанных на изменении критической энергии 
$\gamma_{tr}$ (Рисунок \ref{fig:jump}):

\begin{enumerate} 
  \item Ускорение от энергии инжекции $E_{inj}$ со стационарным значением $\gamma_{\ tr}^{stat}$;
  \item  Плавное увеличение $\gamma_{tr}$ параллельно энергии частиц до пикового значения, коэффициент скольжения $\eta_0$ приобретает минимально возможное значение, приближаясь к нулевому значению;
  \item Переход через стационарное значение критической энергии, при этом $\eta_0$ пересекает нулевое значение для всех частиц;
  \item Плавное восстановление $\gamma_{tr}$ до стационарного значения, также параллельно энергии частиц;
  \item Ускорение до энергии эксперимента со стационарным значением критической энергии $\gamma_{\ tr}^{stat}$.
  \end{enumerate}
  
  \begin{figure}[!htb]
   \includegraphics*[width=.49\columnwidth]{img/fig_01-1}
   \includegraphics*[width=.49\columnwidth]{img/fig_01-2}
   \caption{Схема скачка критической энергии. Синяя линия – фактическая критическая энергия ускорителя $\gamma_{tr}$, красная линия – энергия референсной частицы.}
   \label{fig:jump}
\end{figure}


Состояния 2-3-4 определяют процедуру преодоления $\gamma_{tr}$ скачком. Изменение магнитооптики приводит к зависимости $\gamma_{tr}$, соответствующего смещения рабочей точки $\nu_{x,y}$ (Рисунок \ref{fig:tr}), а также высших порядков коэффициента расширения орбиты $\alpha_1, \alpha_2$ (Рисунок \ref{fig:alpha}).

\begin{figure}[!htb]
   \includegraphics*[width=.49\columnwidth]{img/fig_02-1}
   \includegraphics*[width=.49\columnwidth]{img/fig_02-2}
   \caption{Зависимость критической энергии и рабочей точки от возмущения градиента квадрупольных линз.}
   \label{fig:tr}
\end{figure}

\begin{figure}[!htb]
   \includegraphics*[width=.49\columnwidth]{img/fig_03-1}
   \includegraphics*[width=.49\columnwidth]{img/fig_03-2}
   \caption{Зависимость высших порядков разложения коэффициента расширения орбиты от критической энергии.}
   \label{fig:alpha}
\end{figure}

\section{ВЧ барьерного типа}

\par Для прохождения критической энергии, возможно использование ВЧ барьерного типа (Barrier Bucker RF) \cite{BB_NICA}, \cite{BB_Mal}. (Рисунок \ref{fig:rf})

\begin{figure}[!htb]
  \centering
   \includegraphics*[width=.75\columnwidth]{img/fig_04-1}
   \caption{Нормализированная форма сигнала от ВЧ барьера.}
   \label{fig:rf}
\end{figure}

\begin{equation}
g(\phi)=\left\{\begin{array}{c}
-\operatorname{sign}(\eta),\quad -\pi / h_r \leq \phi \leq 0 \\
\operatorname{sign}(\eta),\quad 0<\phi \leq \pi / h_r \\
0, \quad \text { other }
\end{array}\right.
\label{sign}
\end{equation}

где $\eta$ – коэффициент скольжения (slip-factor), $h_r=\frac{\pi}{\phi_{r}}$ – гармоническое число для отражающего барьера и $\phi_{r}$ – соответствующая фаза.  В Уравнении~$\ref{sign}$ учтено, что при прохождении через критическую энергию, знак $\eta$ меняется и, соответственно, полярность ВЧ барьеров. Для ускорения может быть также приложено дополнительное напряжение в виде меандра с напряжением $V_{acc}=300~\rm{eV}$.

Коэффициенты Фурье-разложения для приведенного прямоугольного сигнала даются выражением \cite{BB_cern}:

\begin{equation}
b_n=\operatorname{sign}{\left(\eta\right)}\frac{2}{n\pi}\left[1-\cos{\left(\frac{n}{h_r}\pi\right)}\right],
\label{b}
\end{equation}

где $n$ – номер гармоники. Для создания плавной формы сигнала, используется сигма-модуляция, сохраняющая симметрию сигнала:

\begin{equation}
\sigma_{m, n}={\rm sinc}^m{\frac{n\pi}{2\left(N+1\right)}},
\label{sigma}
\end{equation}

где $N$ – количество членов гармонического разложения. Таким образом, напряжение n-ой гармоники:

\begin{equation}
V_n=V^{peak}b_n\sigma_{m, n}.
\label{Volt_n}
\end{equation}

На Рисунках \ref{fig:wave} представлены полученные формы сигнала и соответствующие напряжения для гармоник.

В зависимости от относительного смещения от референсной, частицы попадают под влияния ВЧ барьера – в области отражения и испытывают толчок энергии:

\begin{equation}
E_i^\prime=\Delta E_i+\sum_{j=1}^{N} V_j\sin{\left(\omega_{j}\mathrm{\Delta}t_i+\phi_j\right)} 
\label{dE}
\end{equation}

\begin{figure}[!htb]
   \includegraphics*[width=.49\columnwidth]{img/fig_05-1}
   \includegraphics*[width=.49\columnwidth]{img/fig_05-2}
   \caption{Разложение сигнала от ВЧ барьерного типа в ряд Фурье по синусоидальным гармоникам. Слева – форма 
   ВЧ барьеров, справа – амплитуды гармоник в зависимости от частоты для разной ширины отражающего барьера.}
   \label{fig:wave}
\end{figure}

\section{Учёт влияния импедансов}

\par Для учета влияния электромагнитного взаимодействия пучка с окружением вводится понятие импеданса. На продольную динамику основное влияние оказывает импеданс пространственного заряда (Рисунок \ref{fig:signal}) 

\begin{equation}
\frac{Z_{SC}}{n}=\frac{Z_0}{2\beta\gamma^2}\left[1+2\ln{\left(\frac{b}{a}\right)}\right]
\label{sc}
\end{equation}

Для наглядности, приведём напряжение, индуцированное пространственным зарядом, $V_{\mathrm{s.c.\ }}(\phi)$. Уравнение определяется производной от функции распределения $f(\phi)$ в пространстве \cite{Lee_SC}:

\begin{equation}
V_{\rm{S.C.}}\left(\phi\right)=\frac{Z^2h^2g_0Z_0ce}{2R_0\gamma^2}\cdot\frac{\partial\left(N_0f\left(\phi\right)\right)}{\partial\phi}.
\label{V_sc}
\end{equation}

\par Для ВЧ барьерного типа, как будет видно далее из Рисунков \ref{fig:2}, \ref{fig:3}, распределение внутри сепаратрисы равномерное непосредственно вне отражающего барьера. Таким образом, производная слабо отличается от нуля. Значительное напряжение может быть создано только на краях сепаратрисы, где наблюдается изменение градиента в профиле пучка.


\begin{figure}[!htb]
   \includegraphics*[width=.51\columnwidth]{img/fig_06-1}
   \includegraphics*[width=.48\columnwidth]{img/fig_07-1}
   \caption{Слева – импеданс пространственного заряда; справа – Напряжение, создаваемое пространственным 
   зарядом вдоль профиля пучка в продольной плоскости. }
   \label{fig:signal}
\end{figure}

\section{Моделирование}

\par Наиболее опасными с точки зрения разрушения пучка, являются состояния 2-3-4, при которых изменяются параметры ускорителя. С точки зрения динамики, состояния 2 и 4 являются симметричными.
\par Профиль пучка в продольной плоскости равномерный, а энергетический разброс гауссов. Состояние 2 и 4 характерны тем, что коэффициент скольжения для равновесной частицы остается неизменными, а критическая энергия меняется синхронно с энергией пучка в течение порядка $2\times{10}^5$ оборотов. Таким образом, удержание пучка при стационарном значении критической энергии эквивалентно ускоренному движении пучка в структуре с меняющимися параметрами. Как видно на Рисунках 8 профиль пучка смещается к левому барьеру, это связано с тем, что для частиц с положительными $\delta>0$ коэффициент скольжения $\eta_{+\delta}$ больше, чем для частиц с отрицательным $\delta<0$ $\eta_{-\delta}: \eta_{+\delta}>\eta_{-\delta}$. Это видно из Уравнения \ref{eta} и того факта, что $\eta_1<0$. 

\begin{figure}[!htb]
   \includegraphics*[width=.49\columnwidth]{img/fig_08-1}
   \includegraphics*[width=.49\columnwidth]{img/fig_08-2}
   \caption{Фазовая плоскость при удержании пучка внутри ВЧ-барьера. Слева – начальное распределение, справа – распределение после $2\times{10}^5 оборотов$.}
   \label{fig:2}
\end{figure}

\par Состояние 3 – быстрое изменение параметров в течение $6\times{10}^3$ оборотов ($10~\rm{ ms}$). ВЧ-барьеры выключены, чтобы не разрушить пучок. Влияние пространственного заряда наиболее важно в отсутствие барьеров, так как отсутствует внешняя удерживающая сила. Трекинг сделан с учетом описанного выше импеданса пространственного заряда.

\begin{figure}[!h]
   \includegraphics*[width=.49\columnwidth]{img/fig_08-1}
   \includegraphics*[width=.49\columnwidth]{img/fig_08-2}
   \caption{Фазовая плоскость при скачке, ВЧ-барьеры отключены. Слева – начальное распределение, справа – распределение после $6\times{10}^3$ оборотов.}
   \label{fig:3}
\end{figure}

\par За время скачка существенного изменения профиля пучка не произошло. Моделирование выполнено в среде BLonD \cite{blond_mod}, \cite{blond}.

\section{Заключение}

\par Изучена динамика продольного движения вблизи критической энергии в ВЧ барьерного типа, с учётом импеданса пространственного заряда.  
\par Процедура скачкообразного изменения параметров ускорителя является доступным вариантом преодоления критической энергии.

\section{При поддежке}
Это исследование выполнено при поддержке Российского научного фонда №22-42-04419. https://rscf.ru/en/project/22-42-04419/


\ifboolexpr{bool{jacowbiblatex}}
	\bibliography{}
	{
	\begin{thebibliography}{9}
	
	\bibitem{Lee}
	S Y Lee, Accelerator Physics 3rd Edition, \url{https://doi.org/10.1142/8335}
	\bibitem{tr}
	T. Risselada, Gamma Transition Jump Schemes, CAS 1994.
	\bibitem{tr_pip}
	R. Ainsworth at al., Transition Crossing in the Main Injector For PIP-II, FERMILAB-CONF-17-143-AD
	\bibitem{BB_th}
	Hans Stockhorst at al., Beam Cooling at COSY and HESR, ISBN 978-3-95806-127-9
	\bibitem{resonant}
	Yu. V. Senichev, A. N. Chechenin, Theory of “Resonant” Lattices for Synchrotrons with Negative Momentum Compaction 		Factor, Journal of Experimental and Theoretical Physics, 2007, Vol. 105, No. 5, pp. 988–997
	\bibitem{syresin}
	Syresin E.M at al., Formation of Polarized Proton Beams in the NICA Collider-Accelerator Complex DOI: 10.1134/			S1063779621050051
	\bibitem{BB_NICA}
	A. Tribendis and others, Constraction and first test results of the barrier 
	and harmonic RF systems for the NICA collider, IPAC2021, Campinas, SP, Brazil, doi:10.18429/JACoW-IPAC2021-			MOPAB365
	\bibitem{BB_Mal}
	A.M. Malyshev and others, Barrier station RF1 of the NICA collider. 
	Design features and influence on beam dynamics, RuPAC2021, Alushta, Russia, doi:10.18429/JACoW-RuPAC2021-		WEPSC15
	\bibitem{BB_cern}
	Mihaly Vadai, Beam Loss Reduction by Barrier Buckets in the CERN Accelerator Complex, CERN, Geneva, 2021
	\bibitem{Laclare}
	Laclare, J L (ESRF, Grenoble), Coasting beam longitudinal coherent instabilities, CAS - CERN Accelerator School: 5th 		General Accelerator Physics Course, pp.349-384, DOI: 10.5170/CERN-1994-001.349
	\bibitem{Lee_SC}
	J. Wei, S. Y.  Lee, Space Charge Effect at Transition Energy and the Transfer of R.F. System at Top Energy, BNL-41667
	\bibitem{blond_mod}
	P. F. Derwent, Implementation of BLonD for Booster Simulations, Beams doc 8690, 2020
	\bibitem{blond}
	BLonD
	\url{https://blond.web.cern.ch/}

	\end{thebibliography}
}

\end{document}