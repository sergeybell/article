\documentclass[a4paper]{panl}
\usepackage{cite}
\usepackage{wrapfig}
\usepackage{graphicx}
\usepackage{amssymb}
\usepackage{amsfonts}
\usepackage{amsmath}
\usepackage{longtable}
\usepackage{rotating}
\usepackage{lscape}
\usepackage{epsfig}
\usepackage{multirow}


\originalTeX
%\russianTeX
\begin{document}
% Journal sections (see http://pkp.jinr.ru/index.php/PEPAN_LETTERS/about/editorialPolicies#focusAndScope)
%\issuearea{Physics of Elementary Particles and Atomic Nuclei. Theory}
% or in Russian
%\issuearea{ФИЗИКА ЭЛЕМЕНТАРНЫХ ЧАСТИЦ И АТОМНОГО ЯДРА. ТЕОРИЯ}
%
\title{Longitudinal dynamic in NICA Barrier Bucket RF system at transition energy including impedances in BLonD \\ Продольная динамика NICA в ВЧ барьерного типа при критической энергии, включая импедансы в BLonD}
\maketitle
\authors{S.\,Kolokolchikov $^{a,b,}$\footnote{E-mail: sergey.bell13@gmail.com},
Yu.\,Senichev $^{a,b,}$,
A.\,Aksentev $^{a,b,c}$}
\authors{
A.\,Melnikov $^{a,b,d}$,
V.\,Ladygin $^{e}$,
E.\,Syresin $^{e}$}

\authors{С.\, Колокольчиков$^{a,b,}$\footnote{E-mail: sergey.bell13@gmail.com},
Ю.\,Сеничев$^{a,b,}$,
А.\,Аксентьев $^{a,b,c}$}
\authors{
А.\,Мельников $^{a,b,d}$,
В.\,Ладыгин $^{e}$,
Е.\,Сыресин $^{e}$}
\setcounter{footnote}{0}
\from{$^{a}$\,Institute for Nuclear Research RAS, Moscow}
\from{$^{a}$\,Институт ядерных исследований РАН, Москва}
\from{$^{b}$\,Moscow Institute of Physics and Technology, Dolgoprudny}
\from{$^{b}$\,Московский физико-технического института (НИУ), Долгопрудный}
\from{$^{c}$\,Moscow Engineering Physics Institute, Moscow}
\from{$^{c}$\,Московский инженерно-физический институт (НИУ), Москва}
\from{$^{d}$\,Landau Institute for Theoretical Physics, Chernogolovka}
\from{$^{d}$\,Институт теоретической физики им. Л.Д. Ландау, Черноголовка}
\from{$^{e}$\,The Joint Institute for Nuclear Research, Dubna}
\from{$^{e}$\,Объединенный институт ядерных исследований, Дубна}

\begin{abstract}
% Russian translation of the abstract
В статье исследуется влияние импедансов пространственного заряда, а также ВЧ на продольную динамику во время процедуры преодоления критической энергии скачком. Отличительной особенностью является использование ВЧ барьерного типа, в результате чего достигается специфическое распределение пучка в фазовом пространстве, отличное от классического, формируемого гармоническим ВЧ.\
\vspace{0.2cm}

The paper investigates the influence of space charge impedances, as well as RF resonators, on longitudinal dynamics during the procedure of transition energy crossing with a jump. A distinctive feature is the use of Barrier Bucket RF, as a result a specific distribution of the beam in the phase space, different from the classical one formed by harmonic RF.
\end{abstract}
\vspace*{6pt}

\noindent
PACS: 29.20.D−; 29.27.-a

\label{sec:transition}
\section*{Критическая энергия}

\par При рассмотрении продольного движения вводится понятие коэф\-фи\-ци\-ента
расширения орбиты (momentum compaction factor) \cite{lee}:

\begin{equation}
\alpha_c=\frac{1}{R_0} \frac{d R}{d \delta}=\alpha_0+2 \alpha_1 \delta+3 \alpha_2 \delta^2+\cdots \equiv \frac{1}{\gamma_T^2}
\label{alpha}
\end{equation}

и коэффициента скольжения (slip-factor):

\begin{equation}
\eta(\delta)=-\frac{1}{\omega_0} \frac{\Delta \omega}{\delta}=-\left(\eta_0+\eta_1 \delta+\eta_2 \delta^2+\cdots\right),
\label{eta}
\end{equation}

где $\delta$ – разброс по импульсам, $R_{0},R$ – усреднённый радиус референсной и отклоненной на $\delta$ частиц, $\omega, \omega_{0}$ – соответствующие частоты, $\alpha_n, \eta_n$ – n-ые члены разложения, $\gamma_{tr}$ – критическая энергия. Коэффициенты могут быть связаны соотношениями $\eta=\eta_{0}=\alpha_{0}-\frac{1}{\gamma_{0}^{2}}$, $\eta_{1}=\alpha_{1}-\frac{\eta_{0}}{\gamma_{0}^{2}}+\frac{3}{2} \frac{\beta^{2}}{\gamma^{2}}$ . Как видно при определённой энергии референсной частицы -- критической $\gamma = \gamma_{tr}$, коэффициент скольжения принимает нулевое значение $\eta = \eta_{0} = 0$.

\label{sec:jump}
\section*{Скачок критической энергии}

\par Процедура скачка критической энергии применяется для преодоления критической энергии. Таким образом, удается сохранить устойчивое движение пучка в фазовом пространстве. Данный метод применялся на многих установках и описан в работах \cite{jump}, \cite{pip}.

\par Необходимость скачка можно понять, рассмотрев зависимость от $\eta(\delta)\\=\eta_{0}+\eta_{1} \delta+\cdots$, уравнений продольного движения, которые описывают эволюцию частиц в фазовом пространстве \cite{hans}:

\begin{equation}
\begin{aligned}
& \frac{d \tau}{d t}=\eta(\delta) \cdot \frac{h \cdot \Delta E}{\beta^2 \cdot E_0} \\
& \frac{d(\Delta E)}{d t}=\frac{V(\tau)}{T_0}
\end{aligned}
\label{long}
\end{equation}

\par При ускорении, значение коэффициента скольжения $\eta$ приближается к нулю для всех частиц, однако из-за ненулевого разброса по импульсам $\delta$, слагаемое $\eta_1\delta$ начинает быть сравнимо с $\eta_0$ и играет важную роль на динамику вблизи критической энергии. Если не предпринимать никаких мер, то для частиц, преодолевших критическую энергию, знак ко\-эф\-фи\-ци\-ента скольжения меняется. Исходя из Уравнений \ref{long}, видно, что движение в фазовой плоскости становится не устойчивым ведёт к потере пучка. Процедура скачка позволяет, во-первых, в течение поднятия критической энергии, удерживать пучок на расстоянии, достаточном, чтобы все час\-ти\-цы имели один и тот же знак коэффициента скольжения. Во-вторых, о\-бес\-пе\-чить быстрый переход к новому состоянию, где ко\-эф\-фи\-ци\-ент сколь\-же\-ния меняет знак, но для всех частиц снова имеет одинаковый знак. Стабильность обеспечивается сменой полярности у\-дер\-жи\-ва\-ющ\-их ВЧ-барьеров.

 Для коэффициента расширения орбиты может быть получено вы\-ра\-же\-ние \cite{resonant}:
 
\begin{equation}
\alpha=\frac{1}{C}\int_{0}^{C}{\frac{D\left(s\right)}{\rho\left(s\right)}ds}
\label{alpha_c}
\end{equation}

где $D\left(s\right)$– дисперсионная функция, $1/\rho\left(s\right)$ – кривизна орбиты. Для стационарной машины, возможно вариация дисперсионной функции для изменения значения $\alpha$, а соответственно и $\eta$. Например, для NICA, рассматривается возможность создания дополнительного градиента в квадрупольных линзах. Расчёты показывают, что возможно изменение критической энергии $\gamma_{tr}$ со скоростью $d\gamma_{tr}/\ dt\ =\ 8.5\ c^{-1}$ \cite{syresin}.

Можно выделить пять основных состояний продольной динамики, основанных на изменении критической энергии 
$\gamma_{tr}$ (Рисунок \ref{fig:jump}):

\begin{figure}[!h]
   \includegraphics*[width=.49\columnwidth]{img/fig_01-1}
   \includegraphics*[width=.49\columnwidth]{img/fig_01-2}
   \caption{Схема скачка критической энергии. Синяя линия – фактическая критическая энергия ускорителя $\gamma_{tr}$, красная линия – энергия референсной частицы.}
   \label{fig:jump}
\end{figure}

\begin{enumerate} 
  \item Ускорение от энергии инжекции $E_{inj}$ со стационарным значением $\gamma_{\ tr}^{stat}$;
  \item  Плавное увеличение $\gamma_{tr}$ параллельно энергии частиц до пикового значения, коэффициент скольжения $\eta_0$ приобретает минимально возможное значение, приближаясь к нулевому значению;
  \item Переход через стационарное значение критической энергии, при этом $\eta_0$ пересекает нулевое значение для всех частиц;
  \item Плавное восстановление $\gamma_{tr}$ до стационарного значения, также па\-рал\-лель\-но энергии частиц;
  \item Ускорение до энергии эксперимента со стационарным значением критической энергии $\gamma_{\ tr}^{stat}$.
  \end{enumerate}
 
Состояния 2-3-4 определяют процедуру преодоления $\gamma_{tr}$ скачком. Из\-ме\-не\-ние магнитооптики приводит к зависимости $\gamma_{tr}$, соответствующего смещения рабочей точки $\nu_{x,y}$ (Рисунок \ref{fig:tr}), а также высших порядков коэффициента расширения орбиты $\alpha_1, \alpha_2$ (Рисунок \ref{fig:alpha}).

\begin{figure}[!h]
   \includegraphics*[width=.49 \columnwidth]{img/fig_02-1}
   \includegraphics*[width=.49 \columnwidth]{img/fig_02-2}
   \caption{Зависимость критической энергии и рабочей точки от возмущения градиента квадрупольных линз.}
   \label{fig:tr}
\end{figure}

\begin{figure}[!h]
   \includegraphics*[width=.49\columnwidth]{img/fig_03-1}
   \includegraphics*[width=.49\columnwidth]{img/fig_03-2}
   \caption{Зависимость высших порядков разложения коэффициента расширения орбиты от критической энергии.}
   \label{fig:alpha}
\end{figure}

\label{sec:jump}
\section*{ВЧ барьерного типа}

\par Для прохождения критической энергии, возможно использование ВЧ барьерного типа (Barrier Bucker RF) \cite{bb}, \cite{malyshev}. (Рисунок \ref{fig:rf})

\begin{figure}[!h]
  \centering
   \includegraphics*[width=.75\columnwidth]{img/fig_04-1}
   \caption{Нормализированная форма сигнала от ВЧ барьера.}
   \label{fig:rf}
\end{figure}

\begin{equation}
g(\phi)=\left\{\begin{array}{c}
-\operatorname{sign}(\eta),\quad -\pi / h_r \leq \phi \leq 0 \\
\operatorname{sign}(\eta),\quad 0<\phi \leq \pi / h_r \\
0, \quad \text { other }
\end{array}\right.
\label{sign}
\end{equation}

где $\eta$ – коэффициент скольжения (slip-factor), $h_r=\frac{\pi}{\phi_{r}}$ – гармоническое число для отражающего барьера и $\phi_{r}$ – соответствующая фаза.  В У\-рав\-не\-нии~$\ref{sign}$ учтено, что при прохождении через критическую энергию, знак $\eta$ меняется и, соответственно, полярность ВЧ барьеров. Для ускорения может быть также приложено дополнительное напряжение в виде ме\-анд\-ра с нап\-ря\-же\-ни\-ем $V_{acc}=300~\rm{eV}$.

Коэффициенты Фурье-разложения для приведенного прямоугольного сигнала даются выражением \cite{bbcern}:

\begin{equation}
b_n=\operatorname{sign}{\left(\eta\right)}\frac{2}{n\pi}\left[1-\cos{\left(\frac{n}{h_r}\pi\right)}\right],
\label{b}
\end{equation}

где $n$ – номер гармоники. Для создания плавной формы сигнала, используется сигма-модуляция, сохраняющая симметрию сигнала:

\begin{equation}
\sigma_{m, n}={\rm sinc}^m{\frac{n\pi}{2\left(N+1\right)}},
\label{sigma}
\end{equation}

где $N$ – количество членов гармонического разложения. Таким образом, напряжение n-ой гармоники:

\begin{equation}
V_n=V^{peak}b_n\sigma_{m, n}.
\label{Volt_n}
\end{equation}

На Рисунках \ref{fig:wave} представлены полученные формы сигнала и со\-от\-вет\-ству\-ющ\-ие напряжения для гармоник.

В зависимости от относительного смещения от референсной, частицы попадают под влияния ВЧ барьера – в области отражения и испытывают толчок энергии:

\begin{equation}
E_i^\prime=\Delta E_i+\sum_{j=1}^{N} V_j\sin{\left(\omega_{j}\mathrm{\Delta}t_i+\phi_j\right)} 
\label{dE}
\end{equation}

\begin{figure}[!htb]
   \includegraphics*[width=.49\columnwidth]{img/fig_05-1}
   \includegraphics*[width=.49\columnwidth]{img/fig_05-2}
   \caption{Разложение сигнала от ВЧ барьерного типа в ряд Фурье по синусоидальным гармоникам. Слева – форма 
   ВЧ барьеров, справа – амплитуды гармоник в зависимости от частоты для разной ширины отражающего барьера.}
   \label{fig:wave}
\end{figure}

\label{sec:impedance}
\section*{Учёт влияния импедансов}

\par Для учета влияния электромагнитного взаимодействия пучка с ок\-ру\-же\-ни\-ем вводится понятие импеданса. На продольную динамику основное влияние оказывает импеданс пространственного заряда \cite{laclare} (Рисунок \ref{fig:signal}) 

\begin{equation}
\frac{Z_{SC}}{n}=-\frac{Z_0}{2\beta\gamma^2}\left[1+2\ln{\left(\frac{b}{a}\right)}\right]
\label{sc}
\end{equation}

\begin{figure}[!h]
   \includegraphics*[width=.51\columnwidth]{img/fig_06-1}
   \includegraphics*[width=.48\columnwidth]{img/fig_07-1}
   \caption{Слева – импеданс пространственного заряда; справа – Напряжение, создаваемое пространственным 
   зарядом вдоль профиля пучка в продольной плоскости. }
   \label{fig:signal}
\end{figure}

Для наглядности, приведём напряжение, индуцированное про\-стран\-стве\-нным зарядом, $V_{\mathrm{s.c.\ }}(\phi)$. Уравнение определяется производной от функции распределения $f(\phi)$ в пространстве \cite{sc}:

\begin{equation}
V_{\rm{S.C.}}\left(\phi\right)=\frac{Z^2h^2g_0Z_0ce}{2R_0\gamma^2}\cdot\frac{\partial\left(N_0f\left(\phi\right)\right)}{\partial\phi}.
\label{V_sc}
\end{equation}

\par Для ВЧ барьерного типа, как будет видно далее из Рисунков \ref{fig:2}, \ref{fig:3}, распределение внутри сепаратрисы равномерное непосредственно вне от\-ра\-жа\-ющего барьера. Таким образом, производная слабо отличается от нуля. Значительное напряжение может быть создано только на кра\-ях сепаратрисы, где наблюдается изменение градиента в профиле пучка.

\label{sec:tracking}
\section*{Моделирование}

\par Наиболее опасными с точки зрения разрушения пучка, являются со\-сто\-я\-ния 2-3-4, при которых изменяются параметры ускорителя. С точки зрения динамики, состояния 2 и 4 являются симметричными.
\par Профиль пучка в продольной плоскости равномерный, а э\-нер\-ге\-ти\-чес\-кий разброс гауссов. Состояние 2 и 4 характерны тем, что коэффициент скольжения для равновесной частицы остается неизменными, а кри\-ти\-чес\-кая энергия меняется синхронно с энергией пучка в течение порядка $2\times{10}^5$ оборотов. Таким образом, удержание пучка при стационарном значении критической энергии эквивалентно ускоренному движении пуч\-ка в структуре с меняющимися параметрами. Как видно на Рисунках 8 профиль пучка смещается к левому барьеру, это связано с тем, что для частиц с положительными $\delta>0$ коэффициент скольжения $\eta_{+\delta}$ больше, чем для частиц с отрицательным $\delta<0$ $\eta_{-\delta}: \eta_{+\delta}>\eta_{-\delta}$. Это видно из Уравнения \ref{eta} и того факта, что $\eta_1<0$. 

\begin{figure}
   \includegraphics*[width=.49\columnwidth]{img/fig_08-1}
   \includegraphics*[width=.49\columnwidth]{img/fig_08-2}
   \caption{Фазовая плоскость при удержании пучка внутри ВЧ-барьера. Слева – начальное распределение, справа – распределение после $2\times{10}^5 оборотов$.}
   \label{fig:2}
\end{figure}

\par Состояние 3 – быстрое изменение параметров в течение $6\times{10}^3$ о\-бо\-ро\-тов ($10~\rm{ ms}$). ВЧ-барьеры выключены на время скачка, чтобы не разрушить пучок. Влияние пространственного заряда наиболее важно в отсутствие барь\-е\-ров, так как отсутствует внешняя удерживающая сила. Трекинг сделан с учетом описанного выше импеданса пространственного заряда.

\begin{figure}[!h]
   \includegraphics*[width=.49\columnwidth]{img/fig_09-1}
   \includegraphics*[width=.49\columnwidth]{img/fig_09-2}
   \caption{Фазовая плоскость при скачке, ВЧ-барьеры отключены. Слева – начальное распределение, справа – распределение после $6\times{10}^3$ оборотов.}
   \label{fig:3}
\end{figure}

\par За время скачка существенного изменения профиля пучка не про\-и\-зош\-ло. Моделирование выполнено в среде BLonD \cite{blond1}, \cite{blond}.


\label{sec:conc}
\section*{Заключение}
\par Изучена динамика продольного движения вблизи критической э\-нер\-гии в ВЧ барьерного типа, с учётом импеданса пространственного заряда. Процедура скачкообразного изменения параметров ускорителя является доступным вариантом преодоления критической энергии в барьерном ВЧ.

\label{sec:acknowlegments}
\section*{Благодарность}
Это исследование выполнено при поддержке Российского научного фонда №22-42-04419. https://rscf.ru/en/project/22-42-04419/

%\nocite{*}
%\bibliographystyle{pepan}
%\bibliography{maikbibl}
% Bibliography automatically generated via BibTeX (see template_bibtex.tex)

\begin{thebibliography}{1}
\def\selectlanguageifdefined#1{
\expandafter\ifx\csname date#1\endcsname\relax
\else\selectlanguage{#1}\fi}
\providecommand*{\href}[2]{{\small #2}}
\providecommand*{\url}[1]{{\small #1}}
\providecommand*{\BibUrl}[1]{\url{#1}}
\providecommand{\BibAnnote}[1]{}
\providecommand*{\BibEmph}[1]{\emph{#1}}
\ProvideTextCommandDefault{\cyrdash}{\hbox to.8em{--\hss--}}
\providecommand*{\BibDash}{\ifdim\lastskip>0pt\unskip\nobreak\hskip.2em\fi
\cyrdash\hskip.2em\ignorespaces}

%1.	S Y Lee, Accelerator Physics 3rd Edition, https://doi.org/10.1142/8335
%2.	T. Risselada, Gamma Transition Jump Schemes, CAS 1994.
%3.	R. Ainsworth at al., Transition Crossing in the Main Injector For PIP-II, FERMILAB-CONF-17-143-AD
%4.	Hans Stockhorst at al., Beam Cooling at COSY and HESR, ISBN 978-3-95806-127-9
%5.	Yu. V. Senichev, A. N. Chechenin, Theory of “Resonant” Lattices for Synchrotrons with Negative Momentum Compaction Factor, Journal of Experimental and Theoretical Physics, 2007, Vol. 105, No. 5, pp. 988–997
%6.	Syresin E.M at al., Formation of Polarized Proton Beams in the NICA Collider-Accelerator Complex DOI: 10.1134/S1063779621050051
%7.	A. Tribendis and others, Constraction and first test results of the barrier 
%and harmonic RF systems for the NICA collider, IPAC2021, Campinas, SP, Brazil, doi:10.18429/JACoW-IPAC2021-MOPAB365
%8.	A.M. Malyshev and others, Barrier station RF1 of the NICA collider. 
%Design features and influence on beam dynamics, RuPAC2021, Alushta, Russia, doi:10.18429/JACoW-RuPAC2021-WEPSC15
%9.	Mihaly Vadai, Beam Loss Reduction by Barrier Buckets in the CERN Accelerator Complex, CERN, Geneva, 2021
%10.	 Laclare, J L (ESRF, Grenoble), Coasting beam longitudinal coherent instabilities, CAS - CERN Accelerator School: 5th General Accelerator Physics Course, pp.349-384, DOI: 10.5170/CERN-1994-001.349
%11.	 J. Wei, S. Y.  Lee, Space Charge Effect at Transition Energy and the Transfer of R.F. System at Top Energy, BNL-41667
%12.	P. F. Derwent, Implementation of BLonD for Booster Simulations, Beams doc # 8690, 2020
%13.	 BLonD: https://blond.web.cern.ch/

\bibitem{lee}
\selectlanguageifdefined{english}
\BibEmph{S Y Lee} {Accelerator Physics 3rd Edition}~//
  \href{https://doi.org/10.1142/8335}{https://doi.org/10.1142/8335}
  \BibDash
\newblock 1998.

\bibitem{jump}
\selectlanguageifdefined{english}
\BibEmph{T. Risselada} {Gamma Transition Jump Schemes}. \BibDash
\newblock CAS, 1994.

\bibitem{pip}
\selectlanguageifdefined{english}
\BibEmph{R. Ainsworth at al.} Transition Crossing in the Main Injector For PIP-II. \BibDash
\newblock FERMILAB-CONF-17-143-AD.

\bibitem{hans}
\selectlanguageifdefined{english}
\BibEmph{Yu. V. Senichev, A. N. Chechenin} Beam Cooling at COSY and HESR \BibDash
\newblock ISBN 978-3-95806-127-9.

\bibitem{resonant}
\selectlanguageifdefined{english}
\BibEmph{Yu. V. Senichev, A. N. Chechenin} Theory of “Resonant” Lattices for Synchrotrons with Negative Momentum Compaction Factor \BibDash
\newblock JETP \BibDash
\newblock Vol. 105, No. 5, pp. 988–997 \BibDash
\newblock 2007

\bibitem{syresin}
\selectlanguageifdefined{english}
\BibEmph{Syresin E.M at al.} Formation of Polarized Proton Beams in the NICA Collider-Accelerator Complex \BibDash
\newblock Physics of Particles and Nuclei \BibDash
\newblock vol. 52, p. 997–1017 (2021) \BibDash
\newblock DOI: 10.1134/S1063779621050051

\bibitem{bb}
\selectlanguageifdefined{english}
\BibEmph{A. Tribendis at al.} Constraction and first test results of the barrier and harmonic RF systems for the NICA collider \BibDash
\newblock IPAC2021, Campinas, SP, Brazil \BibDash 
\newblock doi:10.18429/JACoW-IPAC2021-MOPAB365

\bibitem{malyshev}
\selectlanguageifdefined{english}
\BibEmph{A.M. Malyshev at al.} Barrier station RF1 of the NICA collider. Design features and influence on beam dynamics \BibDash
\newblock RuPAC2021, Alushta, Russia \BibDash 
\newblock doi:10.18429/JACoW-RuPAC2021-WEPSC15

\bibitem{bbcern}
\selectlanguageifdefined{english}
\BibEmph{Mihaly Vadai} Beam Loss Reduction by Barrier Buckets in the CERN Accelerator Complex \BibDash
\newblock CERN, Geneva \BibDash 
\newblock 2021

\bibitem{laclare}
\selectlanguageifdefined{english}
\BibEmph{Laclare, J L (ESRF, Grenoble)} Coasting beam longitudinal coherent instabilities \BibDash
\newblock CAS - CERN Accelerator School: 5th General Accelerator Physics Course \BibDash 
\newblock pp. 349-384 \BibDash 
\newblock DOI: 10.5170/CERN-1994-001.349

\bibitem{sc}
\selectlanguageifdefined{english}
\BibEmph{J. Wei, S. Y.  Lee} Space Charge Effect at Transition Energy and the Transfer of R.F. System at Top Energy \BibDash
\newblock BNL-41667

\bibitem{blond1}
\selectlanguageifdefined{english}
\BibEmph{P. F. Derwent} Implementation of BLonD for Booster Simulations
 \BibDash
\newblock  Beams doc 8690  \BibDash
\newblock 2020

\bibitem{blond}
\selectlanguageifdefined{english}
\BibEmph{BLonD} https://blond.web.cern.ch/

\end{thebibliography}


\end{document}
